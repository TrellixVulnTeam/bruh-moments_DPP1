\documentclass[10pt,twocolumn]{article}
\usepackage[a4paper,top=1cm,bottom=1cm,left=1cm,right=1cm,marginparwidth=1cm]{geometry}
\title{Phases of a Crypto Bull Market\vspace{-2em}}
\date{}
\begin{document}
% \setlength{\parskip}{10px}%
\setlength{\parindent}{0pt}
\setlength{\footskip}{1.5em}
\maketitle
\begin{abstract}
{\em The paragraph corresponding to the summary must be written with 10-point
letter, in italics. The word {\bf Abstract} must be centered, and separated
three lines from the last author's address. Between the title and the first
author leave a space of two lines.}
    
{\bf TLDR:} ...
\end{abstract}

\section{Bitcoin Fundamentals}

{\bf Advice:} When events should kill and asset but don't, take notice. In 2014,
everything went wrong for Bitcoin. Silk-road, which was Bitcoin's biggest
usecase was taken down and Mt. Gox and subsequent crashes. But Bitcoin did not
die out. 

{\bf Fact:} Huge friction to adopt new technology, no new technology will be
adopted due to 2\% in savings. Bitcoin does not make sense as a payment rail
(will never replace VISA). 

{\bf Reasoning:} Off-shore banking analogy: a \$20-30 trillion market.
Billionaires and S\&P500 companies all have assets spread out with many bank
accounts, such that if ever some of their bank accounts get frozen due to
unexpected court showings, etc, they can still remain solvent. Bitcoin is a
perfect alternative off-shore banking system where assets cannot be arbitrairly
frozen or seized. 

{\bf Info:} Obvious adoption curves. People learning about Bitcoin mostly ended
up becoming a believer. Smart people are getting in and buying. 

{\bf Reasoning:} Dollar depreciation. Fixed emission curve. Tested social
concensus. Hard-forks were attempted and failed. 

{\bf Advice:} Treat Bitcoin like an option. Bitcoin is an asset that routinely
falls 80\%. Expect that sometime in the future, this asset will fall 80\%. 

{\bf Info:} Bitcoin has boom-bust cycles and intrinsic value is network driven.
For example, the more people accept Bitcoin, the more liquid it becomes and thus
the more value it's worth. 

{\bf Advice:} VC mindset was incredibly helpful. You make an investment for 10
years and ignore all short-term volatility. 

{\bf Info:} Most new technology competes on technology, ie features, efficiency,
etc. Early first leaders are often leap-frogged because there is so much
innovation happening and it's not about that first patent or breakthrough, but
rather the 10th patent, etc. Exception is Bitcoin. Bitcoin is comparable to JP
Morgan. We are sure JP Morgan will exist next year. JP Morgan gets found guilty
of money laundering, drug trafficing, etc, every year and gets fined, but they
are not going to get shut down because they are so big. Similarly, JP Morgan
doesn't worry when a new bank offers 5\% less fees because JP Morgan is not
competing on the features but rather on the fundamentals tied to their
longevity. Bitcoin's moat: Anonomous creater and relatively unchanged code that
has survived for a decade. 

{\bf Info:} Core value proposition: it makes no sense trying to optimize Bitcoin
as a payment rail beacuse Bitcoin is a terrible payment rail. There are many
other projects that do much better. Bitcoin's core value stems from that it's
optimized and obselete technology is stable at the protocol, code, and
governance levels. 

{\bf Info:} Separate the asset from the protocol. Example: Wrapped Bitcoin,
Bitcoin existing on the Ethereum blockchain. 

{\bf Reasoning:} Bitcoin is perfect as a use for collateral except for its
volatility, which will decrease with time via increased adoption. 

{\bf Theory:} Bitcoin network may be most secure settlement layer. One example
making Bitcoin network more secure than SWIFT is that US cannot arbitrairly cut
countries off the network (North Korea, Iran). In 5 or 10 years, companies will
be settling their transactions (not necessairly in BTC) on the Bitcoin network. 

\section{Investment Advice} %Investment Advice

{\bf Advice:} Don't blow up. If you have leverage, make sure to not get
liquidated and taken out of the game. What is the amount of risk you're taking
and the possible reward?

{\bf Prediction:} BTC is the public crypto store-of-value pay. BTC blockchain is
the most secure settlement layer. Some proof of stake networks that target a
niche (ex: federalism) for gaming, speed of transactions, etc will also be
successful. 

{\bf Info:} Entire crypto-currency sector follows Metcalfe's law (value of an
asset is equal to $n^2$ for $n$ users) 

{\bf Beware:} Don't run into faulty logic using Metcalfe's law. For example, if
token X's sector is valued at \$1 trillion and token Y is just launched and is
competing in that sector, it is faulty logic to say that if token Y addresses
1\% of that sector, it will be worth \$10 billion. 

{\bf Advice:} Be broad and not concentrated. 

{\bf Info:} In the last portion of the bull-run, alt-coins outperform Bitcoin.
Reasoning: People's risk-tolerance grows and have generated their 5x and are
looking for their next 5x. Last phase of a bull-run is when the worst coins
rising. 

{\bf Advice:} Have basket of high-quality alt-coins, basket of medium quality
alt-coins, and basket of shitcoins. This is a short-term (4 months) bet that
those assets will rise, not as a long-term play. 

{\bf Opinion:} Of the top 100 coins by market-cap, 70-80 are fundamentally
worthless. Note that if there is a fundamental flaw in the project initially,
that does not mean it cannot be fixed. Make sure to look at the competence of
the team and practiability of the vision. 

{\bf Advice:} Do not hold 100 crypto assets, ~20 is good enough (based on
removing 80\%)

{\bf Advice:} Learn how to value cryptos. ie, what's the percentage of a
hardfork on crypto X and the value of the forks, etc, requires deep technical
understanding and what technical information matters. ie, what game-theory
exploits exist in a defi launch. 

\section{Future Innovative Sectors} %Future Innovative Sectors

\subsection{Interoperability across blockchains}
Atomic swaps: Cryptographic signature on two blockchains that without
intermediairy allows transfers of value between blockchains. 

\subsubsection{Layer 0}
{\bf Definition:} Hold and coordinate communication between layer one
blockchains. Meant as an ultimate settlement layer. 

Polkadot, which will allow interoperability across many layer ones. 

\subsubsection{Layer 1}
{\bf Definition:} The blockchain that supports the transactions and
communication of a given currency. 

Ethereum and Bitcoin are layer one. 

\subsubsection{Theories}
{\bf Theory:} Bitcoin will be the ultimate collaterol layer. Faster transactions
with Bitcoin will exist on more efficient blockchains. 

{\bf Theory:} In a world where features are not the diffrentiator (ie for
privacy, can use Bitcoin on Monero blockchain, etc), there are only a few
instances where Bitcoin cannot be used as the main currency of that blockchain.
(1) Monetary policy. Ie, community wanting an asset with a different inflation
policy than Bitcoin. (2) Regulatory status. At somepoint someone will launch a
blockchain that cannot be atomic-swapped and cannot be made anonomous. 

\section{Defi} %DEFI

{\bf Advice} Treat DeFi as hyper-speculative. May make sense to be a staker or
provide liquidity if getting 40\% return, not for a 4\% yield because of
multiple previous hacks. 



\section{Miscallaneous and Further Research}
{\bf Info:} Bitcoin as a solution to custody and rehypothecation

{\bf Beware:} Hard forks define the minority chain to be unstable. Via game
theory, attacks on minority chains that use the same consensus mechanisms as the
main chain can be carried out without sacrifice of material as mining equitment
can still be used to mine on the main chain. 
{\bf Question:} What are the limits of interoperability? 
{\bf Fun fact:} Hard for hedge-funds to hold crypto and multiple security risks
that are not part of their business model exists if they work in crypto. ie,
what are the chances someone will do a \$5 wrench attack?

\section{SECTION}
\subsection{SUBSECTIONS}
\subsubsection{SubSubsection}

Start at 39:20 of https://www.youtube.com/watch?v=6eQqQk8lQwE

\end{document}
\documentclass[10pt,twocolumn]{article}
\usepackage[a4paper,top=1cm,bottom=1cm,left=1cm,right=1cm,marginparwidth=1cm]{geometry}
\title{Phases of a Crypto Bull Market\vspace{-2em}}
\date{}
\begin{document}
\setlength{\parskip}{10px}%
\setlength{\parindent}{0pt}
\setlength{\footskip}{1.5em}
\maketitle
\begin{abstract}
{\em The paragraph corresponding to the summary must be written with 10-point
letter, in italics. The word {\bf Abstract} must be centered, and separated
three lines from the last author's address. Between the title and the first
author leave a space of two lines.}
    
{\bf TLDR:} ...
\end{abstract}

\section{Bitcoin Fundamental Bull-Theses}

{\bf Advice:} When events should kill and asset but don't, take notice. In 2014,
everything went wrong for Bitcoin. Silk-road, which was Bitcoin's biggest
usecase was taken down and Mt. Gox and subsequent crashes. But Bitcoin did not
die out. 

{\bf Fact:} Huge friction to adopt new technology, no new technology will be
adopted due to 2\% in savings. Bitcoin does not make sense as a payment rail
(will never replace VISA). 

{\bf Reasoning:} Off-shore banking analogy: a \$20-30 trillion market.
Billionaires and S\&P500 companies all have assets spread out with many bank
accounts, such that if ever some of their bank accounts get frozen due to
unexpected court showings, etc, they can still remain solvent. Bitcoin is a
perfect alternative off-shore banking system where assets cannot be arbitrairly
frozen or seized. 

{\bf Info:} Obvious adoption curves. People learning about Bitcoin mostly ended
up becoming a believer. Smart people are getting in and buying. 

{\bf Reasoning:} Dollar depreciation. Fixed emission curve



\section{Further Research}
{\bf Info:} Bitcoin as a solution to custody and rehypothecation

\section{SECTIONS, NOTES, CITES, FIGURES, EQUATIONS, TABLES AND REFERENCES (12-point, bold)}
\subsection{SECTIONS AND SUBSECTIONS}
\subsubsection{Subsection}
Start at: 13:22 
https://www.youtube.com/watch?v=6eQqQk8lQwE

\end{document}